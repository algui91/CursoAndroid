\section{Interactuar con otras aplicaciones}
\begin{frame}{Interactuar con otras aplicaciones}
    \begin{block}{}
        Ya hemos visto cómo usar Intents para comunicar actividades de la misma app. Cuando un intent se lanza
        al sistema en lugar de a otra activiy, se usa para iniciar el componente apropiado de una aplicación.\\
        Hay dos tipos de Intents:
    \end{block}
    \begin{itemize}
        \pause
        \item {
            \emph{Explícito:} Para iniciar un componente específico (Un activity concreto).\pause
        }
        \item<1->{
            \emph{Implícito:} Para iniciar cualquier componente que pueda controlar la acción requerida. (Echar una foto).
        }
    \end{itemize}
\end{frame}

\section{Interactuar con otras aplicaciones}
\begin{frame}{Interactuar con otras aplicaciones}
    \begin{block}{}
        Con los intents se puede:
    \end{block}
    \begin{itemize}
        \item {
            Mover de una pantalla a otra dentro de nuestra app.\pause
        }
        \item {
            Mandar al usuario a la pantalla de otra aplicación para realizar una acción.
        }
        \item<3->{
            Cualquier pantalla a la que invoquemos puede devolver un resultado.
        }
        \item<4->{
            Podemos permitir que otras aplicaciones lancen una actividad de la nuestra.
        }
    \end{itemize}
\end{frame}
