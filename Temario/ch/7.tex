\section{Menús}
\begin{frame}{Menús}
    \begin{block}{}
        Veremos cómo crear menús con Action Bar, ya que los menús estándar
        están anticuados. Los tres tipos fundamentales de menús son:
    \end{block}
    \begin{itemize}
        \item {
            \textbf{Options menu y action bar:} Elementos principales del menú.
            Aquí deben ir acciones que tienen un impacto global en la app. (buscar, ajustes, crear mensaje etc)\pause
        }
        \item {
            \textbf{Context menu y contextual action mode:} Menús basados
            en el contexto. Son menús flotantes que aparecen al realizar un “long-click”
            en algún elemento. Las acciones se realizan sobre el contenido
            seleccionado. (Ej. Al seleccionar texto, copiar, pegar...)
        }
        \item<3->{
            \textbf{Popup menu:} Menú que despliega una lista de opciones
            al pulsar en una determinada acción. En la app de Gmail, al pulsar
            el botón de contestar a un mensaje  aparece éste menú (Responder a todos, reenviar etc)
        }
    \end{itemize}
\end{frame}
