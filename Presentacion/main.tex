% Copyright 2004 by Till Tantau <tantau@users.sourceforge.net>.
%
% In principle, this file can be redistributed and/or modified under
% the terms of the GNU Public License, version 2.
%
% However, this file is supposed to be a template to be modified
% for your own needs. For this reason, if you use this file as a
% template and not specifically distribute it as part of a another
% package/program, I grant the extra permission to freely copy and
% modify this file as you see fit and even to delete this copyright
% notice.

\documentclass{beamer}
% Replace the \documentclass declaration above
% with the following two lines to typeset your
% lecture notes as a handout:
%\documentclass{article}
%\usepackage{beamerarticle}


% There are many different themes available for Beamer. A comprehensive
% list with examples is given here:
% http://deic.uab.es/~iblanes/beamer_gallery/index_by_theme.html
% You can uncomment the themes below if you would like to use a different
% one:
%\usetheme{AnnArbor}
%\usetheme{Antibes}
%\usetheme{Bergen}
%\usetheme{Berkeley}
%\usetheme{Berlin}
%\usetheme{Boadilla}
%\usetheme{boxes}
%\usetheme{CambridgeUS}
%\usetheme{Copenhagen}
%\usetheme{Darmstadt}
%\usetheme{default}
%\usetheme{Frankfurt}
\usetheme{Goettingen}
%\usetheme{Hannover}
%\usetheme{Ilmenau}
%\usetheme{JuanLesPins}
%\usetheme{Luebeck}
%\usetheme{Madrid}
%\usetheme{Malmoe}
%\usetheme{Marburg}
%\usetheme{Montpellier}
%\usetheme{PaloAlto}
%\usetheme{Pittsburgh}
%\usetheme{Rochester}
%\usetheme{Singapore}
%\usetheme{Szeged}
%\usetheme{Warsaw}

\usepackage{fontspec} % Allows font customization
\usepackage{hyperref}
\usepackage[english,spanish]{babel}

\title{Curso de programación Android}

% A subtitle is optional and this may be deleted
\subtitle{T-Formación}

\author{Alejandro~Alcalde}
% - Give the names in the same order as the appear in the paper.
% - Use the \inst{?} command only if the authors have different
%   affiliation.

\institute[Estudiante de la ETSIIT] % (optional, but mostly needed)
{
  \href{http://elbauldelprogramador.com}{elbauldelprogramador.com}}
% - Use the \inst command only if there are several affiliations.
% - Keep it simple, no one is interested in your street address.

\date{\today}
% - Either use conference name or its abbreviation.
% - Not really informative to the audience, more for people (including
%   yourself) who are reading the slides online

\subject{Curso Programación Android}
% This is only inserted into the PDF information catalog. Can be left
% out.

% If you have a file called "university-logo-filename.xxx", where xxx
% is a graphic format that can be processed by latex or pdflatex,
% resp., then you can add a logo as follows:

% \pgfdeclareimage[height=0.5cm]{university-logo}{university-logo-filename}
% \logo{\pgfuseimage{university-logo}}

% Delete this, if you do not want the table of contents to pop up at
% the beginning of each subsection:
\AtBeginSubsection[]
{
  \begin{frame}<beamer>{Contenidos}
    \tableofcontents[currentsection,currentsubsection]
  \end{frame}
}

% Let's get started
\begin{document}

\begin{frame}
  \titlepage
\end{frame}

\begin{frame}{Contenidos}
  \tableofcontents
  % You might wish to add the option [pausesections]
\end{frame}

% Section and subsections will appear in the presentation overview
% and table of contents.
\section{Preparando el entorno}

\subsection{Instalar eclipse y el plugin ADT}

\subsubsection{Descargar el SDK de Android y eclipse}

\begin{frame}{Descargar el SDK de Android y eclipse}{}
  \begin{itemize}
  \item {
    Eclipse IDE for Java EE Developers - \href{http://www.eclipse.org/downloads/packages/eclipse-ide-java-ee-developers/keplersr2}{\textit{enlace}}
  }
  \item {
    Descargar el SDK - \href{http://developer.android.com/sdk/index.html\#ExistingIDE}{\textit{enlace}}
  }
  \item{
    Una vez descargado e instalado eclipse, instalar el plugin ADT.
  }
  \end{itemize}
\end{frame}

\subsubsection{Instalar el plugin ADT}

\begin{frame}{Instalar el plugin ADT}{}
  \begin{itemize}
  \item {
    En eclipse ir a Help » Install New Software, Click en Add.
  }
  \item {
    En el cuadro de texto, poner un nombre (Ej. ADT Plugin) y en la url \textit{https://dl-ssl.google.com/android/eclipse/}
  }

    \begin{figure}[h]
        \centering
        \includegraphics[scale=.5]{./img/adt.png}
    \end{figure}

    \item{A continuación, instalar las herramientas de desarrollador}

    \begin{figure}[H]
    \centering
    \includegraphics[scale=.4]{./img/dt.png}
    \end{figure}

  \end{itemize}
\end{frame}

\begin{frame}{Instalar el plugin ADT}{}
  \begin{itemize}
  \item {
    Una vez instalado el plugin y reiniciado eclipse, hay que decir dónde se encuentra el SDK.
  }
  \item {
    Window » Preferences » Android » SDK Location
  }
  \item{
    Instalar una imagen de Android y algunos paquetes extra.
    Window » Android SDK Manager
  }
  \begin{figure}[H]
    \centering
    \includegraphics[scale=.25]{./img/sdkmanager.png}
    \end{figure}
  \end{itemize}
\end{frame}

\subsection{Conceptos básicos Android}

% You can reveal the parts of a slide one at a time
% with the \pause command:
\begin{frame}{Conceptos básicos Android}
  \begin{itemize}
    \item {\textbf{View:} Representa el componente básico en el que se apoyan todos los elementos que construyen una interfaz. Todos los elementos que generan interfaces heredan de la clase \texttt{\href{http://developer.android.com/reference/android/view/View.html}{View}}
    \pause
    }

  \item<2-> {
    \textbf{Activity:} Encargada de mostrar la interfaz de usuario e interactuar con él. Responden a los eventos generados por el usuario (pulsar botones etc). Heredan de la clase \href{http://developer.android.com/reference/android/app/Activity.html}{\texttt{Activity}}.
  }
  \item<3-> { \textbf{Services:} No tienen interfaz visual y se ejecutan en segundo plano, se encargan de realizar tareas que deben continuar ejecutandose cuando nuestra aplicación no está en primer plano. Todos los servicios extienden de la clase \texttt{\href{http://developer.android.com/reference/android/app/Service.html}{Service}}
  }
  \end{itemize}
\end{frame}

\begin{frame}{Conceptos básicos Android}
  \begin{itemize}
  \item{
    \textbf{Content Provider:} Ponen un grupo de datos a disposición de distintas aplicaciones, extienden de la clase ContentProvider para implementar los métodos de la interfaz, pero para acceder a esta interfaz se ha de usar una clase llamada ContentResolver.
    \pause
  }
  \item<2-> {
    \textbf{BroadcastReceiver:} Simplemente reciben un mensaje y reaccionan ante él, extienden de la clase BroadcastReceiver, no tienen interfaz de usuario, pero pueden lanzar Actividades como respuesta a un evento o usar NotificationManager para informar al usuario.
  }
  % or you can use the \uncover command to reveal general
  % content (not just \items):
  \item<3-> {
    \textbf{Intent:} Permite realizar la comunicación y transferencia de datos entre objetos de la clase Activity o Service. También permite iniciar otras Activities o lanzar otras aplicaciones.
  }
  \end{itemize}
\end{frame}

\section{Hola Mundo}

\subsection{Crear el proyecto}

\begin{frame}{Blocks}
\begin{block}{Pasos a realizar}
En eclipse, File » New » Android Application Project. Rellenamos la ventana con los siguientes datos:

\begin{figure}[H]
\centering
\includegraphics[scale=.25]{./img/holamundo.png}
\end{figure}
Todo siguiente hasta crear el proyecto.
\end{block}
\end{frame}

% Placing a * after \section means it will not show in the
% outline or table of contents.
\section*{Qué hemos visto}

\begin{frame}{Qué hemos visto}
  \begin{itemize}
  \item
    Cómo preparar el entorno para desarrollar aplicaciones Android.
  \item
    Conceptos básicos Android.
  \item
    Creación de un proyecto Hola Mundo.
  \end{itemize}
\end{frame}

\begin{frame}{¿Y ahora qué?}
  \begin{itemize}
  \item
    A partir de ahora, trabajaremos sobre ejemplos funcionales, deteniéndonos en las partes de código importantes para explicarlas.
  \end{itemize}
\end{frame}

% All of the following is optional and typically not needed.
\appendix
\section<presentation>*{\appendixname}
\subsection<presentation>*{Bibliografía recomendada}

\begin{frame}[allowframebreaks]
  \frametitle<presentation>{Bibliografía recomendada}

  \begin{thebibliography}{10}

  \beamertemplatebookbibitems
  % Start with overview books.

  \bibitem{ProAnd4}
    Satya~Komatineni.
    \newblock {\em \href{http://www.amazon.es/gp/product/1430239301/ref=as_li_ss_tl?ie=UTF8&camp=3626&creative=24822&creativeASIN=1430239301&linkCode=as2&tag=elbaudelpro-21}{Pro Android 4 - Libro en Amazon}}.
    \newblock Apress 2012.


  \beamertemplatearticlebibitems
  % Followed by interesting articles. Keep the list short.

  \bibitem{developerandroid}
    Developer.android.com
    \newblock Documentación oficial de Android.
    \newblock {\em \href{http://developer.android.com/develop/index.html}{developer.android.com}}
  \end{thebibliography}
\end{frame}

\end{document}


